\documentclass[journal]{IEEEtran}

% *** GRAPHICS RELATED PACKAGES ***
%
\ifCLASSINFOpdf
  % \usepackage[pdftex]{graphicx}
  % declare the path(s) where your graphic files are
  % \graphicspath{{../pdf/}{../jpeg/}}
  % and their extensions so you won't have to specify these with
  % every instance of \includegraphics
  % \DeclareGraphicsExtensions{.pdf,.jpeg,.png}
\else
  % or other class option (dvipsone, dvipdf, if not using dvips). graphicx
  % will default to the driver specified in the system graphics.cfg if no
  % driver is specified.
  % \usepackage[dvips]{graphicx}
  % declare the path(s) where your graphic files are
  % \graphicspath{{../eps/}}
  % and their extensions so you won't have to specify these with
  % every instance of \includegraphics
  % \DeclareGraphicsExtensions{.eps}
\fi

\hyphenation{op-tical net-works semi-conduc-tor}


\begin{document}

\title{Analyze Server Configuration Behavior\\ }


\author{Wasif Altaf,
        Asif Altaf,
        Negar Batenipour\\ Supervisor: Christian Tiefenau}


\maketitle



\IEEEpeerreviewmaketitle



\section{Introduction}
\label{sec:Introduction}

We want to analyze the web server configuration process. For this purpose, the large dataset which contains information about security scores of a web server for over one month has been given to us, qualydump.sql file, and we have analyzed the data and tried to identify patterns and common problems that are related to specific types of web servers and domain groups.\\

In our given dataset which we want to evaluate, we have two different tables, history table and domain table. 
In history table, we have several information such as ID, datein, timespan, rating, category, qualys, and domainhistory. The qualy data is in Jason format for each configuration domain. 
In domain table, we have domain information including different URLs and IDs. \\


In our approach, we have used JASON files and analyzed them. 

In first step, we get the history values against domain keys(id). We have defined the HistoryRow class for sorting the list of entries for a domain and getting the lowest and highest rating. We obtain the history table information: id, datain, timespan, rating, category, qualys, and domainhistory. 

In second step, we initialize JASON queries for the dataset. The best and the worst rating file updates because error and null have been removed from the Database.

The important parameters of JASON data are:\\

- \textbf{protocols}: if SSL is enabled 
\par- \textbf{OCSP stapling}: Online Certificate Status Protocol which \par is known as TLS certificate status request as well, means \par client contacts to a third party to confirm the validity of \par each certificate that it encounters. It is time-consuming and \par has high cost.
\par- \textbf{supportsRc4}: Rivest Cipher 4 is a stream cipher. It is \par remarkable for its simplicity and its speed in software.
\par- \textbf{forward secrecy}: property of secure communication \par protocols in which compromises of long-term keys do \par not compromise past session keys. FS protects past sessions \par against future compromises of secret keys or passwords.
\par- \textbf{heartbleed}:  is a security bug in the OpenSSL \par cryptography library, which is a widely used implementation \par of the Transport Layer Security (TLS) protocol. 
\par- \textbf{poodle}: shows the vulnerability that is increased while we \par want to have high interoperability. 
\par- \textbf{hsts.status (absent,present)}: HTTP Strict Transport \par Security (HSTS)is the web security policy which protects \par websites.
\par- \textbf{hpkp.status}: HTTP Public Key Pinning, is the internet \par security mechanism which prevents impersonation in HTTP \par websites.
\par- \textbf{keysize}: number of bits in a key which is used to control \par the cipher operation, so only the correct key converts \par encrypted text to plain text.
\par- \textbf{algorithms (e.g. RSA 2048 bit, or EC 256bit)}: both of \par them are the public-key cryptosystem. 
\par- \textbf{CA}: is an application test in a scriptless format and a \par collaborate automated testing solution for testing modern \par webs and mobile applications.


\section{Implementation}

\subsection{Methodology Overview}
Subsection text here.

% needed in second column of first page if using \IEEEpubid
%\IEEEpubidadjcol







% if have a single appendix:
%\appendix[Proof of the Zonklar Equations]
% or
%\appendix  % for no appendix heading
% do not use \section anymore after \appendix, only \section*
% is possibly needed

% use appendices with more than one appendix
% then use \section to start each appendix
% you must declare a \section before using any
% \subsection or using \label (\appendices by itself
% starts a section numbered zero.)
%


\appendices
\section{Proof of the First Zonklar Equation}
Appendix one text goes here.

% you can choose not to have a title for an appendix
% if you want by leaving the argument blank
\section{}
Appendix two text goes here.


% use section* for acknowledgment
\section*{Acknowledgment}


The authors would like to thank...


% Can use something like this to put references on a page
% by themselves when using endfloat and the captionsoff option.
\ifCLASSOPTIONcaptionsoff
  \newpage
\fi



% trigger a \newpage just before the given reference
% number - used to balance the columns on the last page
% adjust value as needed - may need to be readjusted if
% the document is modified later
%\IEEEtriggeratref{8}
% The "triggered" command can be changed if desired:
%\IEEEtriggercmd{\enlargethispage{-5in}}

% references section

% can use a bibliography generated by BibTeX as a .bbl file
% BibTeX documentation can be easily obtained at:
% http://mirror.ctan.org/biblio/bibtex/contrib/doc/
% The IEEEtran BibTeX style support page is at:
% http://www.michaelshell.org/tex/ieeetran/bibtex/
%\bibliographystyle{IEEEtran}
% argument is your BibTeX string definitions and bibliography database(s)
%\bibliography{IEEEabrv,../bib/paper}
%
% <OR> manually copy in the resultant .bbl file
% set second argument of \begin to the number of references
% (used to reserve space for the reference number labels box)
\begin{thebibliography}{1}

\bibitem{IEEEhowto:kopka}
H.~Kopka and P.~W. Daly, \emph{A Guide to \LaTeX}, 3rd~ed.\hskip 1em plus
  0.5em minus 0.4em\relax Harlow, England: Addison-Wesley, 1999.

\end{thebibliography}

% biography section
% 
% If you have an EPS/PDF photo (graphicx package needed) extra braces are
% needed around the contents of the optional argument to biography to prevent
% the LaTeX parser from getting confused when it sees the complicated
% \includegraphics command within an optional argument. (You could create
% your own custom macro containing the \includegraphics command to make things
% simpler here.)
%\begin{IEEEbiography}[{\includegraphics[width=1in,height=1.25in,clip,keepaspectratio]{mshell}}]{Michael Shell}
% or if you just want to reserve a space for a photo:





% You can push biographies down or up by placing
% a \vfill before or after them. The appropriate
% use of \vfill depends on what kind of text is
% on the last page and whether or not the columns
% are being equalized.

%\vfill

% Can be used to pull up biographies so that the bottom of the last one
% is flush with the other column.
%\enlargethispage{-5in}



% that's all folks
\end{document}


